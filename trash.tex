
%BJO - Erick, I believe you added the following at the very beginning of the project -- I think it's safe enough to remove it, but you might want to note the references or possibly rework them elsewhere?
%BJO - Moving it here for now.
Other possible statistics:

\begin{itemize}
\item Clustered sequences: estimated total diversity reviewed in \cite{Mehr2012-se}
\item Trees: graph theoretical features \cite{Dunn-Walters2002-cu,Dunn-Walters2004-hv,Mehr2004-ej,Shahaf2008-cc,Budeus2015-ab,Yaari2015-ss}.
Applied to make inferences about data \cite{Steiman-Shimony2006-fm}.
\item Trees + sequence level information for selection estimates \cite{Uduman2014-pb}
\end{itemize}


% %EM I also felt that the following was a bit much before the "in this paper" sentence, but we could recast it here as a motivation for why we wanted a range of statistics, from unprocessed to highly processed.
% In any case, summary statistics are useful for immune receptor repertoire analysis and comparison, for both experimental and simulated repertoires.
% Besides the summaries discussed above, researchers have already derived many other means of quantifying important features of repertoires.
% These summary statistics can range from simple statistics such as the level of GC content, to statistics such as tree shapes which are applied to complex phylogenetic inferences.
% Simple statistics acting on minimally processed sequences are helpful because they are robust to potentially problematic assumptions of inferential methods as well as implementation intricacies.
% For example, in linear regression, comparing inferred slopes is less useful when the relationship of the data is not truly linear, but comparing nonparametric statistics like the sample mean or range is still sensible.
% On the other hand, more intricate statistics give more precise descriptions of the data and can yield more elaborate insights than simple ones.
% Model-based statistics are revealing when inferential model fit is reasonable, and can help reduce noise from sampling or measurement error.
% In our linear regression example, comparing the general trends of points can be less distracted by noise than comparing the sets of points directly, and can differentiate between linear associations unlike the sample mean.


